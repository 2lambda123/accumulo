
% Licensed to the Apache Software Foundation (ASF) under one or more
% contributor license agreements. See the NOTICE file distributed with
% this work for additional information regarding copyright ownership.
% The ASF licenses this file to You under the Apache License, Version 2.0
% (the "License"); you may not use this file except in compliance with
% the License. You may obtain a copy of the License at
%
%     http://www.apache.org/licenses/LICENSE-2.0
%
% Unless required by applicable law or agreed to in writing, software
% distributed under the License is distributed on an "AS IS" BASIS,
% WITHOUT WARRANTIES OR CONDITIONS OF ANY KIND, either express or implied.
% See the License for the specific language governing permissions and
% limitations under the License.

\chapter{Analytics}

Accumulo supports more advanced data processing than simply keeping keys
sorted and performing efficient lookups. Analytics can be developed by using
MapReduce and Iterators in conjunction with Accumulo tables.

\section{MapReduce}

Accumulo tables can be used as the source and destination of MapReduce jobs. To
use an Accumulo table with a MapReduce job (specifically with the new Hadoop API
as of version 0.20), configure the job parameters to use the AccumuloInputFormat
and AccumuloOutputFormat. Accumulo specific parameters can be set via these
two format classes to do the following:
\begin{itemize}
\item{Authenticate and provide user credentials for the input}
\item{Restrict the scan to a range of rows}
\item{Restrict the input to a subset of available columns}
\end{itemize}

\subsection{Mapper and Reducer classes}

To read from an Accumulo table create a Mapper with the following class
parameterization and be sure to configure the AccumuloInputFormat.

\small
\begin{verbatim}
class MyMapper extends Mapper<Key,Value,WritableComparable,Writable> {
    public void map(Key k, Value v, Context c) {
        // transform key and value data here
    }
}
\end{verbatim}
\normalsize

To write to an Accumulo table, create a Reducer with the following class
parameterization and be sure to configure the AccumuloOutputFormat. The key
emitted from the Reducer identifies the table to which the mutation is sent. This
allows a single Reducer to write to more than one table if desired. A default table
can be configured using the AccumuloOutputFormat, in which case the output table
name does not have to be passed to the Context object within the Reducer.

\small
\begin{verbatim}
class MyReducer extends Reducer<WritableComparable, Writable, Text, Mutation> {

    public void reduce(WritableComparable key, Iterable<Text> values, Context c) {
        
        Mutation m;
        
        // create the mutation based on input key and value
        
        c.write(new Text("output-table"), m);
    }
}
\end{verbatim}
\normalsize

The Text object passed as the output should contain the name of the table to which
this mutation should be applied. The Text can be null in which case the mutation
will be applied to the default table name specified in the AccumuloOutputFormat
options.

\subsection{AccumuloInputFormat options}

\small
\begin{verbatim}
Job job = new Job(getConf());
AccumuloInputFormat.setInputInfo(job,
        "user",
        "passwd".getBytes(),
        "table",
        new Authorizations());

AccumuloInputFormat.setZooKeeperInstance(job, "myinstance",
        "zooserver-one,zooserver-two");
\end{verbatim}

\Large
\textbf{Optional settings:}
\normalsize

To restrict Accumulo to a set of row ranges:

\small
\begin{verbatim}
ArrayList<Range> ranges = new ArrayList<Range>();
// populate array list of row ranges ...
AccumuloInputFormat.setRanges(job, ranges);
\end{verbatim}
\normalsize

To restrict Accumulo to a list of columns:

\small
\begin{verbatim}
ArrayList<Pair<Text,Text>> columns = new ArrayList<Pair<Text,Text>>();
// populate list of columns
AccumuloInputFormat.fetchColumns(job, columns);
\end{verbatim}
\normalsize

To use a regular expression to match row IDs:

\small
\begin{verbatim}
AccumuloInputFormat.setRegex(job, RegexType.ROW, "^.*");
\end{verbatim}
\normalsize

\subsection{AccumuloOutputFormat options}

\small
\begin{verbatim}
boolean createTables = true;
String defaultTable = "mytable";

AccumuloOutputFormat.setOutputInfo(job,
        "user",
        "passwd".getBytes(),
        createTables,
        defaultTable);

AccumuloOutputFormat.setZooKeeperInstance(job, "myinstance",
        "zooserver-one,zooserver-two");
\end{verbatim}

\Large
\textbf{Optional Settings:}
\normalsize

\small
\begin{verbatim}
AccumuloOutputFormat.setMaxLatency(job, 300000); // milliseconds
AccumuloOutputFormat.setMaxMutationBufferSize(job, 50000000); // bytes
\end{verbatim}
\normalsize

An example of using MapReduce with Accumulo can be found at\\
accumulo/docs/examples/README.mapred

\section{Combiners}

Many applications can benefit from the ability to aggregate values across common
keys. This can be done via Combiner iterators and is similar to the Reduce step in
MapReduce. This provides the ability to define online, incrementally updated
analytics without the overhead or latency associated with batch-oriented
MapReduce jobs.

All that is needed to aggregate values of a table is to identify the fields over which
values will be grouped, insert mutations with those fields as the key, and configure
the table with a combining iterator that supports the summarizing operation
desired.

The only restriction on an combining iterator is that the combiner developer
should not assume that all values for a given key have been seen, since new
mutations can be inserted at anytime. This precludes using the total number of
values in the aggregation such as when calculating an average, for example.

\subsection{Feature Vectors}

An interesting use of combining iterators within an Accumulo table is to store
feature vectors for use in machine learning algorithms. For example, many
algorithms such as k-means clustering, support vector machines, anomaly detection,
etc. use the concept of a feature vector and the calculation of distance metrics to
learn a particular model. The columns in an Accumulo table can be used to efficiently
store sparse features and their weights to be incrementally updated via the use of an
combining iterator.

\section{Statistical Modeling}

Statistical models that need to be updated by many machines in parallel could be
similarly stored within an Accumulo table. For example, a MapReduce job that is
iteratively updating a global statistical model could have each map or reduce worker
reference the parts of the model to be read and updated through an embedded
Accumulo client.

Using Accumulo this way enables efficient and fast lookups and updates of small
pieces of information in a random access pattern, which is complementary to
MapReduce's sequential access model.

