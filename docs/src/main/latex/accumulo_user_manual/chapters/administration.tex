
% Licensed to the Apache Software Foundation (ASF) under one or more
% contributor license agreements. See the NOTICE file distributed with
% this work for additional information regarding copyright ownership.
% The ASF licenses this file to You under the Apache License, Version 2.0
% (the "License"); you may not use this file except in compliance with
% the License. You may obtain a copy of the License at
%
%     http://www.apache.org/licenses/LICENSE-2.0
%
% Unless required by applicable law or agreed to in writing, software
% distributed under the License is distributed on an "AS IS" BASIS,
% WITHOUT WARRANTIES OR CONDITIONS OF ANY KIND, either express or implied.
% See the License for the specific language governing permissions and
% limitations under the License.

\chapter{Administration}

\section{Hardware}

Because we are running essentially two or three systems simultaneously layered
across the cluster: HDFS, Accumulo and MapReduce, it is typical for hardware to
consist of 4 to 8 cores, and 8 to 32 GB RAM. This is so each running process can have
at least one core and 2 - 4 GB each.

One core running HDFS can typically keep 2 to 4 disks busy, so each machine may
typically have as little as 2 x 300GB disks and as much as 4 x 1TB or 2TB disks.

It is possible to do with less than this, such as with 1u servers with 2 cores and 4GB
each, but in this case it is recommended to only run up to two processes per
machine - i.e. DataNode and TabletServer or DataNode and MapReduce worker but
not all three. The constraint here is having enough available heap space for all the
processes on a machine.

\section{Network}

Accumulo communicates via remote procedure calls over TCP/IP for both passing
data and control messages. In addition, Accumulo uses HDFS clients to
communicate with HDFS. To achieve good ingest and query performance, sufficient
network bandwidth must be available between any two machines.

\section{Installation}
Choose a directory for the Accumulo installation. This directory will be referenced
by the environment variable \texttt{\$ACCUMULO\_HOME}. Run the following:

\small
\begin{verbatim}
$ tar xzf accumulo-1.5.0-bin.tar.gz    # unpack to subdirectory
$ mv accumulo-1.5.0 $ACCUMULO_HOME # move to desired location
\end{verbatim}
\normalsize

Repeat this step at each machine within the cluster. Usually all machines have the
same \texttt{\$ACCUMULO\_HOME}.

\section{Dependencies}
Accumulo requires HDFS and ZooKeeper to be configured and running
before starting. Password-less SSH should be configured between at least the
Accumulo master and TabletServer machines. It is also a good idea to run Network
Time Protocol (NTP) within the cluster to ensure nodes' clocks don't get too out of
sync, which can cause problems with automatically timestamped data. 

\section{Configuration}

Accumulo is configured by editing several Shell and XML files found in
\texttt{\$ACCUMULO\_HOME/conf}. The structure closely resembles Hadoop's configuration
files.

\subsection{Edit conf/accumulo-env.sh}

Accumulo needs to know where to find the software it depends on. Edit accumulo-env.sh 
and specify the following:

\begin{enumerate}
\item{Enter the location of the installation directory of Accumulo for \texttt{\$ACCUMULO\_HOME}}
\item{Enter your system's Java home for \texttt{\$JAVA\_HOME}}
\item{Enter the location of Hadoop for \texttt{\$HADOOP\_PREFIX}}
\item{Choose a location for Accumulo logs and enter it for \texttt{\$ACCUMULO\_LOG\_DIR}}
\item{Enter the location of ZooKeeper for \texttt{\$ZOOKEEPER\_HOME}}
\end{enumerate}

By default Accumulo TabletServers are set to use 1GB of memory. You may change
this by altering the value of \texttt{\$ACCUMULO\_TSERVER\_OPTS}. Note the syntax is that of
the Java JVM command line options. This value should be less than the physical
memory of the machines running TabletServers.

There are similar options for the master's memory usage and the garbage collector
process. Reduce these if they exceed the physical RAM of your hardware and
increase them, within the bounds of the physical RAM, if a process fails because of
insufficient memory.

Note that you will be specifying the Java heap space in accumulo-env.sh. You should
make sure that the total heap space used for the Accumulo tserver and the Hadoop
DataNode and TaskTracker is less than the available memory on each slave node in
the cluster. On large clusters, it is recommended that the Accumulo master, Hadoop
NameNode, secondary NameNode, and Hadoop JobTracker all be run on separate
machines to allow them to use more heap space. If you are running these on the
same machine on a small cluster, likewise make sure their heap space settings fit
within the available memory.

\subsection{Native Map}

The tablet server uses a data structure called a MemTable to store sorted key/value
pairs in memory when they are first received from the client. When a minor compaction
occurs, this data structure is written to HDFS. The MemTable will default to using
memory in the JVM but a JNI version, called the native map, can be used to significantly
speed up performance by utilizing the memory space of the native operating system. The
native map also avoids the performance implications brought on by garbage collection
in the JVM by causing it to pause much less frequently.

32-bit and 64-bit Linux versions of the native map ship with the Accumulo dist package.
For other operating systems, the native map can be built from the codebase in two ways-
from maven or from the Makefile.

\begin{enumerate}
\item{Build from maven using the following command: \texttt{mvn clean package -Pnative.}}
\item{Build from the c++ source by running \texttt{make} in the \texttt{\$ACCUMULO\_HOME/server/src/main/c++} directory.}
\end{enumerate}

After building the native map from the source, you will find the artifact in
\texttt{\$ACCUMULO\_HOME/lib/native.} Upon starting up, the tablet server will look
in this directory for the map library. If the file is renamed or moved from its
target directory, the tablet server may not be able to find it.

\subsection{Cluster Specification}

On the machine that will serve as the Accumulo master:

\begin{enumerate}
\item{Write the IP address or domain name of the Accumulo Master to the\\\texttt{\$ACCUMULO\_HOME/conf/masters} file.}
\item{Write the IP addresses or domain name of the machines that will be TabletServers in\\\texttt{\$ACCUMULO\_HOME/conf/slaves}, one per line.}
\end{enumerate}

Note that if using domain names rather than IP addresses, DNS must be configured
properly for all machines participating in the cluster. DNS can be a confusing source
of errors.

\subsection{Accumulo Settings}
Specify appropriate values for the following settings in\\
\texttt{\$ACCUMULO\_HOME/conf/accumulo-site.xml} :

\small
\begin{verbatim}
<property>
    <name>instance.zookeeper.host</name>
    <value>zooserver-one:2181,zooserver-two:2181</value>
    <description>list of zookeeper servers</description>
</property>
\end{verbatim}
\normalsize

This enables Accumulo to find ZooKeeper. Accumulo uses ZooKeeper to coordinate
settings between processes and helps finalize TabletServer failure.


\small
\begin{verbatim}
<property>
    <name>instance.secret</name>
    <value>DEFAULT</value>
</property>
\end{verbatim}
\normalsize

The instance needs a secret to enable secure communication between servers. Configure your
secret and make sure that the \texttt{accumulo-site.xml} file is not readable to other users.

Some settings can be modified via the Accumulo shell and take effect immediately, but
some settings require a process restart to take effect. See the configuration documentation
(available on the monitor web pages) for details.

\subsection{Deploy Configuration}

Copy the masters, slaves, accumulo-env.sh, and if necessary, accumulo-site.xml
from the\\\texttt{\$ACCUMULO\_HOME/conf/} directory on the master to all the machines
specified in the slaves file.

\section{Initialization}

Accumulo must be initialized to create the structures it uses internally to locate
data across the cluster. HDFS is required to be configured and running before
Accumulo can be initialized.

Once HDFS is started, initialization can be performed by executing\\
\texttt{\$ACCUMULO\_HOME/bin/accumulo init} . This script will prompt for a name
for this instance of Accumulo. The instance name is used to identify a set of tables
and instance-specific settings. The script will then write some information into
HDFS so Accumulo can start properly.

The initialization script will prompt you to set a root password. Once Accumulo is
initialized it can be started.

\section{Running}

\subsection{Starting Accumulo}

Make sure Hadoop is configured on all of the machines in the cluster, including
access to a shared HDFS instance. Make sure HDFS and ZooKeeper are running.
Make sure ZooKeeper is configured and running on at least one machine in the
cluster.
Start Accumulo using the \texttt{bin/start-all.sh} script.

To verify that Accumulo is running, check the Status page as described under
\emph{Monitoring}. In addition, the Shell can provide some information about the status of
tables via reading the metadata tables.

\subsection{Stopping Accumulo}

To shutdown cleanly, run \texttt{bin/stop-all.sh} and the master will orchestrate the
shutdown of all the tablet servers. Shutdown waits for all minor compactions to finish, so it may
take some time for particular configurations.

\subsection{Adding a Node}

Update your \texttt{\$ACCUMULO\_HOME/conf/slaves} (or \texttt{\$ACCUMULO\_CONF\_DIR/slaves}) file to account for the addition.

\begin{verbatim}
$ACCUMULO_HOME/bin/accumulo admin start <host(s)> {<host> ...}
\end{verbatim}

Alternatively, you can ssh to each of the hosts you want to add and run 
\texttt{\$ACCUMULO\_HOME/bin/start-here.sh}.

Make sure the host in question has the new configuration, or else the tablet 
server won't start; at a minimum this needs to be on the host(s) being added, 
but in practice it's good to ensure consistent configuration across all nodes.

\subsection{Decomissioning a Node}

If you need to take a node out of operation, you can trigger a graceful shutdown of a tablet 
server. Accumulo will automatically rebalance the tablets across the available tablet servers.

\begin{verbatim}
$ACCUMULO_HOME/bin/accumulo admin stop <host(s)> {<host> ...}
\end{verbatim}

Alternatively, you can ssh to each of the hosts you want to remove and run 
\texttt{\$ACCUMULO\_HOME/bin/stop-here.sh}.

Be sure to update your \texttt{\$ACCUMUL\_HOME/conf/slaves} (or \texttt{\$ACCUMULO\_CONF\_DIR/slaves}) file to 
account for the removal of these hosts. Bear in mind that the monitor will not re-read the 
slaves file automatically, so it will report the decomissioned servers as down; it's 
recommended that you restart the monitor so that the node list is up to date.

\section{Monitoring}

The Accumulo Master provides an interface for monitoring the status and health of
Accumulo components. This interface can be accessed by pointing a web browser to\\
\texttt{http://accumulomaster:50095/status}

\section{Tracing}
It can be difficult to determine why some operations are taking longer
than expected. For example, you may be looking up items with very low
latency, but sometimes the lookups take much longer. Determining the
cause of the delay is difficult because the system is distributed, and
the typical lookup is fast.

Accumulo has been instrumented to record the time that various
operations take when tracing is turned on. The fact that tracing is
enabled follows all the requests made on behalf of the user throughout
the distributed infrastructure of accumulo, and across all threads of
execution.

These time spans will be inserted into the \texttt{trace} table in
Accumulo. You can browse recent traces from the Accumulo monitor
page. You can also read the \texttt{trace} table directly like any
other table.

The design of Accumulo's distributed tracing follows that of
\href{http://research.google.com/pubs/pub36356.html}{Google's Dapper}.

\subsection{Tracers}
To collect traces, Accumulo needs at least one server listed in
\\\texttt{\$ACCUMULO\_HOME/conf/tracers}. The server collects traces
from clients and writes them to the \texttt{trace} table. The Accumulo
user that the tracer connects to Accumulo with can be configured with
the following properties

\begin{verbatim}
trace.user
trace.token.property.password
\end{verbatim}

\subsection{Instrumenting a Client}
Tracing can be used to measure a client operation, such as a scan, as
the operation traverses the distributed system. To enable tracing for
your application call

\begin{verbatim}
DistributedTrace.enable(instance, new ZooReader(instance), hostname, "myApplication");
\end{verbatim}

Once tracing has been enabled, a client can wrap an operation in a trace.

\begin{verbatim}
Trace.on("Client Scan");
BatchScanner scanner = conn.createBatchScanner(...);
// Configure your scanner
for (Entry entry : scanner) {
}
Trace.off();
\end{verbatim}

Additionally, the user can create additional Spans within a Trace.
\begin{verbatim}
Trace.on("Client Update");
...
Span readSpan = Trace.start("Read");
...
readSpan.stop();
...
Span writeSpan = Trace.start("Write");
...
writeSpan.stop();
Trace.off();
\end{verbatim}

Like Dapper, Accumulo tracing supports user defined annotations to associate additional data with a Trace.
\begin{verbatim}
...
int numberOfEntriesRead = 0;
Span readSpan = Trace.start("Read");
// Do the read, update the counter
...
readSpan.data("Number of Entries Read", String.valueOf(numberOfEntriesRead));
\end{verbatim}

Some client operations may have a high volume within your
application. As such, you may wish to only sample a percentage of
operations for tracing. As seen below, the CountSampler can be used to
help enable tracing for 1-in-1000 operations
\begin{verbatim}
Sampler sampler = new CountSampler(1000);
...
if (sampler.next()) {
  Trace.on("Read");
}
...
Trace.offNoFlush();
\end{verbatim}

It should be noted that it is safe to turn off tracing even if it
isn't currently active. The Trace.offNoFlush() should be used if the
user does not wish to have Trace.off() block while flushing trace
data.

\subsection{Viewing Collected Traces}
To view collected traces, use the "Recent Traces" link on the Monitor
UI. You can also programmatically access and print traces using the
\texttt{TraceDump} class.

\subsection{Tracing from the Shell}
You can enable tracing for operations run from the shell by using the
\texttt{trace on} and \texttt{trace off} commands.

\begin{verbatim}
root@test test> trace on
root@test test> scan
a b:c []    d
root@test test> trace off
Waiting for trace information
Waiting for trace information
Trace started at 2013/08/26 13:24:08.332
Time  Start  Service@Location       Name
 3628+0      shell@localhost shell:root
    8+1690     shell@localhost scan
    7+1691       shell@localhost scan:location
    6+1692         tserver@localhost startScan
    5+1692           tserver@localhost tablet read ahead 6
\end{verbatim}

\section{Logging}
Accumulo processes each write to a set of log files. By default these are found under\\
\texttt{\$ACCUMULO/logs/}.

\section{Recovery}

In the event of TabletServer failure or error on shutting Accumulo down, some
mutations may not have been minor compacted to HDFS properly. In this case,
Accumulo will automatically reapply such mutations from the write-ahead log
either when the tablets from the failed server are reassigned by the Master (in the
case of a single TabletServer failure) or the next time Accumulo starts (in the event of
failure during shutdown).

Recovery is performed by asking a tablet server to sort the logs so that tablets can easily find their missing
updates. The sort status of each file is displayed on
Accumulo monitor status page. Once the recovery is complete any
tablets involved should return to an ``online" state. Until then those tablets will be
unavailable to clients.

The Accumulo client library is configured to retry failed mutations and in many
cases clients will be able to continue processing after the recovery process without
throwing an exception.

