
% Licensed to the Apache Software Foundation (ASF) under one or more
% contributor license agreements. See the NOTICE file distributed with
% this work for additional information regarding copyright ownership.
% The ASF licenses this file to You under the Apache License, Version 2.0
% (the "License"); you may not use this file except in compliance with
% the License. You may obtain a copy of the License at
%
%     http://www.apache.org/licenses/LICENSE-2.0
%
% Unless required by applicable law or agreed to in writing, software
% distributed under the License is distributed on an "AS IS" BASIS,
% WITHOUT WARRANTIES OR CONDITIONS OF ANY KIND, either express or implied.
% See the License for the specific language governing permissions and
% limitations under the License.

\chapter{Troubleshooting}

\section{Logs}

Q. The tablet server does not seem to be running!? What happened?

Accumulo is a distributed system.  It is supposed to run on remote
equipment, across hundreds of computers.  Each program that runs on
these remote computers writes down events as they occur, into a local
file. By default, this is defined in
\texttt{\$ACCUMULO\_HOME}/conf/accumule-env.sh as ACCUMULO\_LOG\_DIR.

A. Look in the \texttt{\$ACCUMULO\_LOG\_DIR}/tserver*.log file.  Specifically, check the end of the file.

Q. The tablet server did not start and the debug log does not exists!  What happened?

When the individual programs are started, the stdout and stderr output
of these programs are stored in ``.out'' and ``.err'' files in
\texttt{\$ACCUMULO\_LOG\_DIR}.  Often, when there are missing configuration
options, files or permissions, messages will be left in these files.

A. Probably a start-up problem.  Look in \texttt{\$ACCUMULO\_LOG\_DIR}/tserver*.err

\section{Monitor}

Q. Accumulo is not working, what's wrong?

There's a small web server that collects information about all the
components that make up a running Accumulo instance. It will highlight
unusual or unexpected conditions.

A. Point your browser to the monitor (typically the master host, on port 50095).  Is anything red or yellow?

Q. My browser is reporting connection refused, and I cannot get to the monitor

The monitor program's output is also written to .err and .out files in
the \texttt{\$ACCUMULO\_LOG\_DIR}. Look for problems in this file if the
\texttt{\$ACCUMULO\_LOG\_DIR/monitor*.log} file does not exist.

A. The monitor program is probably not running.  Check the log files for errors.

Q. My browser hangs trying to talk to the monitor.

Your browser needs to be able to reach the monitor program.  Often
large clusters are firewalled, or use a VPN for internal
communications. You can use SSH to proxy your browser to the cluster,
or consult with your system administrator to gain access to the server
from your browser.

It is sometimes helpful to use a text-only browser to sanity-check the
monitor while on the machine running the monitor:

\begingroup\fontsize{8pt}{8pt}\selectfont\begin{verbatim}
  $ links http://localhost:50095
\end{verbatim}\endgroup

A. Verify that you are not firewalled from the monitor if it is running on a remote host.

Q. The monitor responds, but there are no numbers for tservers and tables.  The summary page says the master is down.

The monitor program gathers all the details about the master and the
tablet servers through the master. It will be mostly blank if the
master is down.

A. Check for a running master.

\section{HDFS}

Accumulo reads and writes to the Hadoop Distributed File System.
Accumulo needs this file system available at all times for normal operations.

Q. Accumulo is having problems ``getting a block blk\_1234567890123.'' How do I fix it?

This troubleshooting guide does not cover HDFS, but in general, you
want to make sure that all the datanodes are running and an fsck check
finds the file system clean:

\begingroup\fontsize{8pt}{8pt}\selectfont\begin{verbatim}
  $ hadoop fsck /accumulo
\end{verbatim}\endgroup

You can use:

\begingroup\fontsize{8pt}{8pt}\selectfont\begin{verbatim}
  $ hadoop fsck /accumulo/path/to/corrupt/file -locations -blocks -files
\end{verbatim}\endgroup

to locate the block references of individual corrupt files and use those
references to search the name node and individual data node logs to determine which 
servers those blocks have been assigned and then try to fix any underlying file
system issues on those nodes.

On a larger cluster, you may need to increase the number of Xceivers

\begingroup\fontsize{8pt}{8pt}\selectfont\begin{verbatim}
  <property>
    <name>dfs.datanode.max.xcievers</name>
    <value>4096</value>
  </property>
\end{verbatim}\endgroup

A. Verify HDFS is healthy, check the datanode logs.

\section{Zookeeper}

Q. \texttt{accumulo init} is hanging.  It says something about talking to zookeeper.

Zookeeper is also a distributed service.  You will need to ensure that
it is up.  You can run the zookeeper command line tool to connect to
any one of the zookeeper servers:

\begingroup\fontsize{8pt}{8pt}\selectfont\begin{verbatim}
  $ zkCli.sh -server zoohost
...
[zk: zoohost:2181(CONNECTED) 0] 
\end{verbatim}\endgroup

It is important to see the word \texttt{CONNECTED}!  If you only see
\texttt{CONNECTING} you will need to diagnose zookeeper errors.

A. Check to make sure that zookeeper is up, and that
\texttt{\$ACCUMULO\_HOME/conf/accumulo-site.xml} has been pointed to
your zookeeper server(s).

Q. Zookeeper is running, but it does not say \texttt{CONNECTED}

Zookeeper processes talk to each other to elect a leader.  All updates
go through the leader and propagate to a majority of all the other
nodes.  If a majority of the nodes cannot be reached, zookeeper will
not allow updates.  Zookeeper also limits the number connections to a
server from any other single host.  By default, this limit can be as small as 10 
and can be reached in some everything-on-one-machine test configurations.

You can check the election status and connection status of clients by
asking the zookeeper nodes for their status.  You connect to zookeeper
and ask it with the four-letter ``stat'' command:

\begingroup\fontsize{8pt}{8pt}\selectfont\begin{verbatim}
$ nc zoohost 2181
stat
Zookeeper version: 3.4.5-1392090, built on 09/30/2012 17:52 GMT
Clients:
 /127.0.0.1:58289[0](queued=0,recved=1,sent=0)
 /127.0.0.1:60231[1](queued=0,recved=53910,sent=53915)

Latency min/avg/max: 0/5/3008
Received: 1561459
Sent: 1561592
Connections: 2
Outstanding: 0
Zxid: 0x621a3b
Mode: standalone
Node count: 22524
$
\end{verbatim}\endgroup


A. Check zookeeper status, verify that it has a quorum, and has not exceeded maxClientCnxns.

Q. My tablet server crashed!  The logs say that it lost it's zookeeper lock.

Tablet servers reserve a lock in zookeeper to maintain their ownership
over the tablets that have been assigned to them.  Part of their
responsibility for keeping the lock is to send zookeeper a keep-alive
message periodically.  If the tablet server fails to send a message in
a timely fashion, zookeeper will remove the lock and notify the tablet
server.  If the tablet server does not receive a message from
zookeeper, it will assume its lock has been lost, too.  If a tablet
server loses its lock, it kills itself: everything assumes it is dead
already.

A. Investigate why the tablet server did not send a timely message to
zookeeper.

\subsection{Keeping the tablet server lock}

Q. My tablet server lost its lock.  Why?

The primary reason a tablet server loses its lock is that it has been pushed into swap.

A large java program (like the tablet server) may have a large portion
of its memory image unused.  The operation system will favor pushing
this allocated, but unused memory into swap so that the memory can be
re-used as a disk buffer.  When the java virtual machine decides to
access this memory, the OS will begin flushing disk buffers to return that
memory to the VM.  This can cause the entire process to block long
enough for the zookeeper lock to be lost.

A. Configure your system to reduce the kernel parameter ``swappiness'' from the default (60) to zero.

Q. My tablet server lost its lock, and I have already set swappiness to
zero.  Why?

Be careful not to over-subscribe memory.  This can be easy to do if
your accumulo processes run on the same nodes as hadoop's map-reduce
framework.  Remember to add up:

\begin{itemize}
\item{size of the JVM for the tablet server}
\item{size of the in-memory map, if using the native map implementation}
\item{size of the JVM for the data node}
\item{size of the JVM for the task tracker}
\item{size of the JVM times the maximum number of mappers and reducers}
\item{size of the kernel and any support processes}
\end{itemize}

If a 16G node can run 2 mappers and 2 reducers, and each can be 2G,
then there is only 8G for the data node, tserver, task tracker and OS.

A. Reduce the memory footprint of each component until it fits comfortably.

Q. My tablet server lost its lock, swappiness is zero, and my node has lots of unused memory!

The JVM memory garbage collector may fall behind and cause a
``stop-the-world'' garbage collection. On a large memory virtual
machine, this collection can take a long time.  This happens more
frequently when the JVM is getting low on free memory.  Check the logs
of the tablet server.  You will see lines like this:

\begingroup\fontsize{8pt}{8pt}\selectfont\begin{verbatim}
2013-06-20 13:43:20,607 [tabletserver.TabletServer] DEBUG: gc ParNew=0.00(+0.00) secs
    ConcurrentMarkSweep=0.00(+0.00) secs freemem=1,868,325,952(+1,868,325,952) totalmem=2,040,135,680
\end{verbatim}\endgroup

When ``freemem'' becomes small relative to the amount of memory
needed, the JVM will spend more time finding free memory than
performing work.  This can cause long delays in sending keep-alive
messages to zookeeper.

A. Ensure the tablet server JVM is not running low on memory.

\section{Tools}

The accumulo script can be used to run classes from the command line.
This section shows how a few of the utilities work, but there are many
more.

There's a class that will examine an accumulo storage file and print
out basic metadata.  

\begingroup\fontsize{8pt}{8pt}\selectfont\begin{verbatim}
$ ./bin/accumulo org.apache.accumulo.core.file.rfile.PrintInfo /accumulo/tables/1/default_tablet/A000000n.rf
2013-07-16 08:17:14,778 [util.NativeCodeLoader] INFO : Loaded the native-hadoop library
Locality group         : <DEFAULT>
        Start block          : 0
        Num   blocks         : 1
        Index level 0        : 62 bytes  1 blocks
        First key            : 288be9ab4052fe9e span:34078a86a723e5d3:3da450f02108ced5 [] 1373373521623 false
        Last key             : start:13fc375709e id:615f5ee2dd822d7a [] 1373373821660 false
        Num entries          : 466
        Column families      : [waitForCommits, start, md major compactor 1, md major compactor 2, md major compactor 3,
                                 bringOnline, prep, md major compactor 4, md major compactor 5, md root major compactor 3,
                                 minorCompaction, wal, compactFiles, md root major compactor 4, md root major compactor 1,
                                 md root major compactor 2, compact, id, client:update, span, update, commit, write,
                                 majorCompaction]

Meta block     : BCFile.index
      Raw size             : 4 bytes
      Compressed size      : 12 bytes
      Compression type     : gz

Meta block     : RFile.index
      Raw size             : 780 bytes
      Compressed size      : 344 bytes
      Compression type     : gz
\end{verbatim}\endgroup

When trying to diagnose problems related to key size, the PrintInfo tool can provide a histogram of the individual key sizes:

\begingroup\fontsize{8pt}{8pt}\selectfont\begin{verbatim}
$ ./bin/accumulo org.apache.accumulo.core.file.rfile.PrintInfo --histogram /accumulo/tables/1/default_tablet/A000000n.rf
...
Up to size      count      %-age
         10 :        222  28.23%
        100 :        244  71.77%
       1000 :          0   0.00%
      10000 :          0   0.00%
     100000 :          0   0.00%
    1000000 :          0   0.00%
   10000000 :          0   0.00%
  100000000 :          0   0.00%
 1000000000 :          0   0.00%
10000000000 :          0   0.00%
\end{verbatim}\endgroup

Likewise, PrintInfo will dump the key-value pairs and show you the contents of the RFile:

\begingroup\fontsize{8pt}{8pt}\selectfont\begin{verbatim}
$ ./bin/accumulo org.apache.accumulo.core.file.rfile.PrintInfo --dump /accumulo/tables/1/default_tablet/A000000n.rf
row columnFamily:columnQualifier [visibility] timestamp deleteFlag -> Value
...
\end{verbatim}\endgroup

Q. Accumulo is not showing me any data!

A. Do you have your auths set so that it matches your visibilities?

Q. What are my visibilities?

A. Use ``PrintInfo'' on a representative file to get some idea of the visibilities in the underlying data.

Note that the use of PrintInfo is an administrative tool and can only
by used by someone who can access the underlying Accumulo data. It
does not provide the normal access controls in Accumulo.

If you would like to backup, or otherwise examine the contents of Zookeeper, there are commands to dump and load to/from XML.

\begingroup\fontsize{8pt}{8pt}\selectfont\begin{verbatim}
$ ./bin/accumulo org.apache.accumulo.server.util.DumpZookeeper --root /accumulo >dump.xml
$ ./bin/accumulo org.apache.accumulo.server.util.RestoreZookeeper --overwrite < dump.xml
\end{verbatim}\endgroup

Q. How can I get the information in the monitor page for my cluster monitoring system?

A. Use GetMasterStats:

\begingroup\fontsize{8pt}{8pt}\selectfont\begin{verbatim}
$ ./bin/accumulo org.apache.accumulo.test.GetMasterStats | grep Load
 OS Load Average: 0.27
\end{verbatim}\endgroup

Q. The monitor page is showing an offline tablet.  How can I find out which tablet it is?

A. Use FindOfflineTablets:

\begingroup\fontsize{8pt}{8pt}\selectfont\begin{verbatim}
$ ./bin/accumulo org.apache.accumulo.server.util.FindOfflineTablets
2<<@(null,null,localhost:9997) is UNASSIGNED  #walogs:2
\end{verbatim}\endgroup

Here's what the output means:

\begin{enumerate}
\item{\texttt{2<<} This is the tablet from (-inf, +inf) for the
  table with id 2.  ``tables -l'' in the shell will show table ids for
  tables.}
\item{@(null, null, localhost:9997)} Location information.  The
  format is \texttt{@(assigned, hosted, last)}.  In this case, the
  tablet has not been assigned, is not hosted anywhere, and was once
  hosted on localhost.
\item{\#walogs:2} The number of write-ahead logs that this tablet requires for recovery.
\end{enumerate}

An unassigned tablet with write-ahead logs is probably waiting for
logs to be sorted for efficient recovery.

Q. How can I be sure that the metadata tables are up and consistent?

A. \texttt{CheckForMetadataProblems} will verify the start/end of
every tablet matches, and the start and stop for the table is empty:

\begingroup\fontsize{8pt}{8pt}\selectfont\begin{verbatim}
$ ./bin/accumulo org.apache.accumulo.server.util.CheckForMetadataProblems -u root --password
Enter the connection password: 
All is well for table !0
All is well for table 1
\end{verbatim}\endgroup

Q. My hadoop cluster has lost a file due to a NameNode failure.  How can I remove the file?

A. There's a utility that will check every file reference and ensure
that the file exists in HDFS.  Optionally, it will remove the
reference:

\begingroup\fontsize{8pt}{8pt}\selectfont\begin{verbatim}
$ ./bin/accumulo org.apache.accumulo.server.util.RemoveEntriesForMissingFiles -u root --password
Enter the connection password: 
2013-07-16 13:10:57,293 [util.RemoveEntriesForMissingFiles] INFO : File /accumulo/tables/2/default_tablet/F0000005.rf
 is missing
2013-07-16 13:10:57,296 [util.RemoveEntriesForMissingFiles] INFO : 1 files of 3 missing
\end{verbatim}\endgroup

Q. I have many entries in zookeeper for old instances I no longer need.  How can I remove them?

A. Use CleanZookeeper:

\begingroup\fontsize{8pt}{8pt}\selectfont\begin{verbatim}
$ ./bin/accumulo org.apache.accumulo.server.util.CleanZookeeper
\end{verbatim}\endgroup

This command will not delete the instance pointed to by the local \texttt{conf/accumulo-site.xml} file.

Q. I need to decommission a node.  How do I stop the tablet server on it?

A. Use the admin command:

\begingroup\fontsize{8pt}{8pt}\selectfont\begin{verbatim}
$ ./bin/accumulo admin stop hostname:9997
2013-07-16 13:15:38,403 [util.Admin] INFO : Stopping server 12.34.56.78:9997
\end{verbatim}\endgroup

Q. I cannot login to a tablet server host, and the tablet server will not shut down.  How can I kill the server?

A. Sometimes you can kill a ``stuck'' tablet server by deleting it's lock in zookeeper:

\begingroup\fontsize{8pt}{8pt}\selectfont\begin{verbatim}
$ ./bin/accumulo org.apache.accumulo.server.util.TabletServerLocks --list
                  127.0.0.1:9997 TSERV_CLIENT=127.0.0.1:9997
$ ./bin/accumulo org.apache.accumulo.server.util.TabletServerLocks -delete 127.0.0.1:9997
$ ./bin/accumulo org.apache.accumulo.server.util.TabletServerLocks -list
                  127.0.0.1:9997             null
\end{verbatim}\endgroup

You can find the master and instance id for any accumulo instances using the same zookeeper instance:

\begingroup\fontsize{8pt}{8pt}\selectfont\begin{verbatim}
$ ./bin/accumulo org.apache.accumulo.server.util.ListInstances
INFO : Using ZooKeepers localhost:2181

 Instance Name       | Instance ID                          | Master                        
---------------------+--------------------------------------+-------------------------------
              "test" | 6140b72e-edd8-4126-b2f5-e74a8bbe323b |                127.0.0.1:9999
\end{verbatim}\endgroup

\section{System Metadata Tables}
\label{sec:metadata}

Accumulo tracks information about tables in metadata tables. The metadata for
most tables is contained within the metadata table in the accumulo namespace,
while metadata for that table is contained in the root table in the accumulo
namespace. The root table is composed of a single tablet, which does not
split, so it is also called the root tablet. Information about the root
table, such as its location and write-ahead logs, are stored in ZooKeeper.

Let's create a table and put some data into it:

\begingroup\fontsize{8pt}{8pt}\selectfont\begin{verbatim}
shell> createtable test
shell> tables -l
accumulo.metadata    =>        !0
accumulo.root        =>        +r
test                 =>         2
trace                =>         1
shell> insert a b c d
shell> flush -w
\end{verbatim}\endgroup

Now let's take a look at the metadata for this table:

\begingroup\fontsize{8pt}{8pt}\selectfont\begin{verbatim}
shell> table accumulo.metadata
shell> scan -b 3; -e 3<
3< file:/default_tablet/F000009y.rf []    186,1
3< last:13fe86cd27101e5 []    127.0.0.1:9997
3< loc:13fe86cd27101e5 []    127.0.0.1:9997
3< log:127.0.0.1+9997/0cb7ce52-ac46-4bf7-ae1d-acdcfaa97995 []    127.0.0.1+9997/0cb7ce52-ac46-4bf7-ae1d-acdcfaa97995|6
3< srv:dir []    /default_tablet
3< srv:flush []    1
3< srv:lock []    tservers/127.0.0.1:9997/zlock-0000000001$13fe86cd27101e5
3< srv:time []    M1373998392323
3< ~tab:~pr []    \x00
\end{verbatim}\endgroup

Let's decode this little session:

\begin{enumerate}
\item{\texttt{scan -b 3; -e 3<}\\
  Every tablet gets its own row. Every row starts with the table id followed by
  ``;'' or ``<'', and followed by the end row split point for that tablet.}
\item{\texttt{file:/default\_tablet/F000009y.rf [] 186,1}\\
  File entry for this tablet.  This tablet contains a single file reference. The
  file is ``/accumulo/tables/3/default\_tablet/F000009y.rf''.  It contains 1
  key/value pair, and is 186 bytes long.}
\item{\texttt{last:13fe86cd27101e5 []    127.0.0.1:9997}\\
  Last location for this tablet.  It was last held on 127.0.0.1:9997, and the
  unique tablet server lock data was ``13fe86cd27101e5''. The default balancer
  will tend to put tablets back on their last location.}
\item{\texttt{loc:13fe86cd27101e5 []    127.0.0.1:9997}\\
  The current location of this tablet.}
\item{\texttt{log:127.0.0.1+9997/0cb7ce52-ac46-4bf7-ae1d-acdcfaa97995 []    127.0. ...}\\
  This tablet has a reference to a single write-ahead log. This file can be found in\\
  /accumulo/wal/127.0.0.1+9997/0cb7ce52-ac46-4bf7-ae1d-acdcfaa97995. The value
  of this entry could refer to multiple files. This tablet's data is encoded as
  ``6'' within the log.}
\item{\texttt{srv:dir []    /default\_tablet}\\
  Files written for this tablet will be placed into
  /accumulo/tables/3/default\_tablet.}
\item{\texttt{srv:flush []    1}\\
  Flush id.  This table has successfully completed the flush with the id of
  ``1''.}
\item{\texttt{srv:lock []    tservers/127.0.0.1:9997/zlock-0000000001\$13fe86cd27101e5}\\
  This is the lock information for the tablet holding the present lock.  This
  information is checked against zookeeper whenever this is updated, which
  prevents a metadata update from a tablet server that no longer holds its
  lock.}
\item{\texttt{srv:time []    M1373998392323} }
\item{\texttt{\textasciitilde{}tab:\textasciitilde{}pr []    \textbackslash{}x00}\\
  The end-row marker for the previous tablet (prev-row).  The first byte
  indicates the presence of a prev-row.  This tablet has the range (-inf, +inf),
  so it has no prev-row (or end row).}
\end{enumerate}

Besides these columns, you may see:

\begin{enumerate}
\item{\texttt{rowId future:zooKeeperID location} Tablet has been assigned to a tablet, but not yet loaded.}
\item{\texttt{\textasciitilde{}del:filename} When a tablet server is done use a file, it will create a delete marker in the appropriate metadata table, unassociated with any tablet.  The garbage collector will remove the marker, and the file, when no other reference to the file exists.}
\item{\texttt{\textasciitilde{}blip:txid} Bulk-Load In Progress marker}
\item{\texttt{rowId loaded:filename} A file has been bulk-loaded into this tablet, however the bulk load has not yet completed on other tablets, so this is marker prevents the file from being loaded multiple times.}
\item{\texttt{rowId !cloned} A marker that indicates that this tablet has been successfully cloned.}
\item{\texttt{rowId splitRatio:ratio} A marker that indicates a split is in progress, and the files are being split at the given ratio.}
\item{\texttt{rowId chopped} A marker that indicates that the files in the tablet do not contain keys outside the range of the tablet.}
\item{\texttt{rowId scan} A marker that prevents a file from being removed while there are still active scans using it.}

\end{enumerate}

\section{Simple System Recovery}

Q. One of my Accumulo processes died. How do I bring it back?

The easiest way to bring all services online for an Accumulo instance is to run the ``start-all.sh`` script.

\begingroup\fontsize{8pt}{8pt}\selectfont\begin{verbatim}
  $ bin/start-all.sh
\end{verbatim}\endgroup

This process will check the process listing, using ``jps`` on each host before attempting to restart a service on the given host.
Typically, this check is sufficient except in the face of a hung/zombie process. For large clusters, it may be
undesirable to ssh to every node in the cluster to ensure that all hosts are running the appropriate processes and ``start-here.sh`` may be of use.

\begingroup\fontsize{8pt}{8pt}\selectfont\begin{verbatim}
  $ ssh host_with_dead_process
  $ bin/start-here.sh
\end{verbatim}\endgroup

``start-here.sh`` should be invoked on the host which is missing a given process. Like start-all.sh, it will start all
necessary processes that are not currently running, but only on the current host and not cluster-wide. Tools such as ``pssh`` or 
``pdsh`` can be used to automate this process.

``start-server.sh`` can also be used to start a process on a given host; however, it is not generally recommended for
users to issue this directly as the ``start-all.sh`` and ``start-here.sh`` scripts provide the same functionality with
more automation and are less prone to user error.

A. Use ``start-all.sh`` or ``start-here.sh``.

Q. My process died again. Should I restart it via ``cron`` or tools like ``supervisord``?

A. A repeatedly dying Accumulo process is a sign of a larger problem. Typically these problems are due to a
misconfiguration of Accumulo or over-saturation of resources. Blind automation of any service restart inside of Accumulo
is generally an undesirable situation as it is indicative of a problem that is being masked and ignored. Accumulo
processes should be stable on the order of months and not require frequent restart.


\section{Advanced System Recovery}

\subsection{HDFS Failure}
Q. I had disasterous HDFS failure.  After bringing everything back up, several tablets refuse to go online.

Data written to tablets is written into memory before being written into indexed files.  In case the server
is lost before the data is saved into a an indexed file, all data stored in memory is first written into a
write-ahead log (WAL).  When a tablet is re-assigned to a new tablet server, the write-ahead logs are read to
recover any mutations that were in memory when the tablet was last hosted.

If a write-ahead log cannot be read, then the tablet is not re-assigned.  All it takes is for one of
the blocks in the write-ahead log to be missing.  This is unlikely unless multiple data nodes in HDFS have been
lost.

A. Get the WAL files online and healthy.  Restore any data nodes that may be down.

Q. How do find out which tablets are offline?

A. Use ``accumulo admin checkTablets''

\begingroup\fontsize{8pt}{8pt}\selectfont\begin{verbatim}
  $ bin/accumulo admin checkTablets
\end{verbatim}\endgroup

Q. I lost three data nodes, and I'm missing blocks in a WAL.  I don't care about data loss, how
can I get those tablets online?

See the discussion in section~\ref{sec:metadata}, which shows a typical metadata table listing.  
The entries with a column family of ``log'' are references to the WAL for that tablet. 
If you know what WAL is bad, you can find all the references with a grep in the shell:

\begingroup\fontsize{8pt}{8pt}\selectfont\begin{verbatim}
shell> grep 0cb7ce52-ac46-4bf7-ae1d-acdcfaa97995
3< log:127.0.0.1+9997/0cb7ce52-ac46-4bf7-ae1d-acdcfaa97995 []    127.0.0.1+9997/0cb7ce52-ac46-4bf7-ae1d-acdcfaa97995|6
\end{verbatim}\endgroup

A. You can remove the WAL references in the metadata table.

\begingroup\fontsize{8pt}{8pt}\selectfont\begin{verbatim}
shell> grant -u root Table.WRITE -t accumulo.metadata
shell> delete 3< log 127.0.0.1+9997/0cb7ce52-ac46-4bf7-ae1d-acdcfaa97995
\end{verbatim}\endgroup

Note: the colon (``:'') is omitted when specifying the ``row cf cq'' for the delete command.

The master will automatically discover the tablet no longer has a bad WAL reference and will
assign the tablet.  You will need to remove the reference from all the tablets to get them 
online.


Q. The metadata (or root) table has references to a corrupt WAL.

This is a much more serious state, since losing updates to the metadata table will result
in references to old files which may not exist, or lost references to new files, resulting
in tablets that cannot be read, or large amounts of data loss.

The best hope is to restore the WAL by fixing HDFS data nodes and bringing the data back online.
If this is not possible, the best approach is to re-create the instance and bulk import all files from
the old instance into a new tables.

A complete set of instructions for doing this is outside the scope of this guide,
but the basic approach is:

\begin{itemize}
 \item Use ``tables -l'' in the shell to discover the table name to table id mapping
 \item Stop all accumulo processes on all nodes
 \item Move the accumulo directory in HDFS out of the way:
\begingroup\fontsize{8pt}{8pt}\selectfont\begin{verbatim}
 $ hadoop fs -mv /accumulo /corrupt
\end{verbatim}\endgroup
 \item Re-initalize accumulo
 \item Recreate tables, users and permissions
 \item Import the directories under \texttt{/corrupt/tables/<id>} into the new instance
\end{itemize}

Q. One or more HDFS Files under /accumulo/tables are corrupt

Accumulo maintains multiple references into the tablet files in the METADATA
table and within the tablet server hosting the file, this makes it difficult to
reliably just remove those references.

The directory structure in HDFS for tables will follow the general structure:

\small
\begin{verbatim}
  /accumulo
  /accumulo/tables/
  /accumulo/tables/!0
  /accumulo/tables/!0/default_tablet/A000001.rf
  /accumulo/tables/!0/t-00001/A000002.rf
  /accumulo/tables/1
  /accumulo/tables/1/default_tablet/A000003.rf
  /accumulo/tables/1/t-00001/A000004.rf
  /accumulo/tables/1/t-00001/A000005.rf
  /accumulo/tables/2/default_tablet/A000006.rf
  /accumulo/tables/2/t-00001/A000007.rf
\end{verbatim}
\normalsize

If files under /accumulo/tables are corrupt, the best course of action is to
recover those files in hdsf see the section on HDFS. Once these recovery efforts
have been exhausted, the next step depends on where the missing file(s) are
located. Different actions are required when the bad files are in Accumulo data
table files or if they are metadata table files.

{\bf Data File Corruption}

When an Accumulo data file is corrupt, the most reliable way to restore Accumulo
operations is to replace the missing file with an “empty” file so that
references to the file in the METADATA table and within the tablet server
hosting the file can be resolved by Accumulo. An empty file can be created using
the CreateEmpty utiity:

\begingroup\fontsize{8pt}{8pt}\selectfont\begin{verbatim}
  $accumulo org.apache.accumulo.core.file.rfile.CreateEmpty /path/to/empty/file/empty.rf
\end{verbatim}\endgroup

The process is to delete the corrupt file and then move the empty file into its
place (The generated empty file can be copied and used multiple times if necessary and does not need
to be regenerated each time)

\begingroup\fontsize{8pt}{8pt}\selectfont\begin{verbatim}
  $hadoop fs –rm /accumulo/tables/corrupt/file/thename.rf; \
  hadoop fs -mv /path/to/empty/file/empty.rf /accumulo/tables/corrupt/file/thename.rf
\end{verbatim}\endgroup

{\bf Metadata File Corruption}

If the corrupt files are metadata files, see \ref{sec:metadata} (under the path
\begin{verbatim}/accumulo/tables/!0\end{verbatim}) then you will need to rebuild
the metadata table by initializing a new instance of Accumulo and then importing
all of the existing data into the new instance.  This is the same procedure as
recovering from a zookeeper failure (see \ref{ZooKeeper Failure}, except that
you will have the benefit of having the existing user and table authorizations
that are maintained in zookeeper.

You can use the DumpZookeeper utility to save this information for reference
before creating the new instance.  You will not be able to use RestoreZookeeper
because the table names and references are likely to be different between the
original and the new instances, but it can serve as a reference.

A. If the files cannot be recovered, replace corrupt data files with a empty
rfiles to allow references in the metadata table and in the tablet servers to be
resolved. Rebuild the metadata table if the corrupt files are metadata files.

\subsection{ZooKeeper Failure}
Q. I lost my ZooKeeper quorum (hardware failure), but HDFS is still intact. How can I recover my Accumulo instance?

ZooKeeper, in addition to its lock-service capabilities, also serves to bootstrap an Accumulo
instance from some location in HDFS. It contains the pointers to the root tablet in HDFS which
is then used to load the Accumulo metadata tablets, which then loads all user tables. ZooKeeper
also stores all namespace and table configuration, the user database, the mapping of table IDs to 
table names, and more across Accumulo restarts.

Presently, the only way to recover such an instance is to initialize a new instance and import all
of the old data into the new instance. The easiest way to tackle this problem is to first recreate
the mapping of table ID to table name and then recreate each of those tables in the new instance. 
Set any necessary configuration on the new tables and add some split points to the tables to close 
the gap between how many splits the old table had and no splits.

The directory structure in HDFS for tables will follow the general structure:

\small
\begin{verbatim}
  /accumulo
  /accumulo/tables/
  /accumulo/tables/1
  /accumulo/tables/1/default_tablet/A000001.rf
  /accumulo/tables/1/t-00001/A000002.rf
  /accumulo/tables/1/t-00001/A000003.rf
  /accumulo/tables/2/default_tablet/A000004.rf
  /accumulo/tables/2/t-00001/A000005.rf
\end{verbatim}
\normalsize

For each table, make a new directory that you can move (or copy if you have the HDFS space to do so)
all of the rfiles for a given table into. For example, to process the table with an ID of ``1``, make a new directory, 
say ``/new-table-1`` and then copy all files from ``/accumulo/tables/1/*/*.rf`` into that directory. Additionally,
make a directory, ``/new-table-1-failures``, for any failures during the import process. Then, issue the import
command using the Accumulo shell into the new table, telling Accumulo to not re-set the timestamp:

\small
\begin{verbatim}
user@instance new_table> importdirectory /new-table-1 /new-table-1-failures false
\end{verbatim}
\normalsize

Any RFiles which were failed to be loaded will be placed in ``/new-table-1-failures``. Rfiles that were successfully
imported will no longer exist in ``/new-table-1``. For failures, move them back to the import directory and retry
the ``importdirectory`` command.

It is \textbf{extremely} important to note that this approach may introduce stale data back into
the tables. For a few reasons, RFiles may exist in the table directory which are candidates for deletion but have
not yet been deleted. Additionally, deleted data which was not compacted away, but still exists in write-ahead logs if
the original instance was somehow recoverable, will be re-introduced in the new instance. Table splits and merges
(which also include the deleteRows API call on TableOperations, are also vulnerable to this problem. This process should
\textbf{not} be used if these are unacceptable risks. It is possible to try to re-create a view of the ``accumulo.metadata``
table to prune out files that are candidates for deletion, but this is a difficult task that also may not be entirely accurate.

Likewise, it is also possible that data loss may occur from write-ahead log (WAL) files which existed on the old table but
were not minor-compacted into an RFile. Again, it may be possible to reconstruct the state of these WAL files to
replay data not yet in an RFile; however, this is a difficult task and is not implemented in any automated fashion.

A. The ``importdirectory`` shell command can be used to import RFiles from the old instance into a newly created instance,
but extreme care should go into the decision to do this as it may result in reintroduction of stale data or the
omission of new data.

\section{File Naming Conventions}

Q. Why are files named like they are? Why do some start with ``C'' and others with ``F''?

A. The file names give you a basic idea for the source of the file.

The base of the filename is a base-36 unique number. All filenames in accumulo are coordinated 
with a counter in zookeeper, so they are always unique, which is useful for debugging.

The leading letter gives you an idea of how the file was created:

\begin{itemize}
 \item F - Flush: entries in memory were written to a file (Minor Compaction)
 \item M - Merging compaction: entries in memory were combined with the smallest file to create one new file
 \item C - Several files, but not all files, were combined to produce this file (Major Compaction)
 \item A - All files were compacted, delete entries were dropped
 \item I - Bulk import, complete, sorted index files. Always in a directory starting with "b-"
\end{itemize}

This simple file naming convention allows you to see the basic structure of the files from just 
their filenames, and reason about what should be happening to them next, just
by scanning their entries in the metadata tables.

For example, if you see multiple files with ``M'' prefixes, the tablet is, or was, up against it's
maximum file limit, so it began merging memory updates with files to keep the file count reasonable.  This
slows down ingest performance, so knowing there are many files like this tells you that the system
is struggling to keep up with ingest vs the compaction strategy which reduces the number of files.

